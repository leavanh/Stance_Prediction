\begin{abstract}\label{abstract}

Recent advancements in Natural Language Processing (NLP) have wide-ranging implications, including in politics, where language models can be used to give voters useful information as they navigate a complicated political landscape. For example, a model could assist users by predicting which parties align with statements chosen by the user, aiding in informed decision-making. This thesis explores the effectiveness of incorporating information from political party manifestos to enhance the accuracy of large language models that predict party stances. The proposed model takes user-generated statements and relevant passages from party manifestos as inputs to determine the political party that is most likely to agree with the statement. The thesis evaluates the performance of different semantic search techniques for identifying relevant passages and various input patterns for conveying the additional information to the model. Two NLP models, ELECTRA and BERT, are compared to determine their effectiveness in this task. The study found that incorporating information from party manifestos did not substantially improve the model's ability to predict party stances, although some semantic search techniques provide better results than others. Further research is needed to determine the effectiveness of different input patterns, models, and hyperparameters, and to investigate how various other context types can be incorporated into the model.

\end{abstract}
