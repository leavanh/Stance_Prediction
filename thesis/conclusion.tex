\chapter{Conclusion}\label{conclusion}

In order to increase the accuracy of large-scale language models for predicting party stance positions, this thesis looked into the possibility of using relevant information from political party manifestos. The use of NLP models in the field of politics has gained significant interest, and this work contributes to the development of such models.

In order to predict whether political parties would agree with a statement, the models proposed in this study use user-generated statements and relevant sentences from party manifestos as inputs. Sentences from the 2021 party manifestos were chosen using eight different semantic search techniques, namely BM25, SBERT, BERT-Flow (and SBERT-BERT-Flow), Whitening (and SBERT-Whitening), SBERT-WK, and IS-BERT. The effectiveness of seven different input patterns and two language models, ELECTRA \citep{clark2020electra} and BERT \citep{devlin2018bert}, was investigated.

The results show that although including information from political party manifestos could occasionally increase the model's capacity to predict party stances, the improvement was not substantial. The performance of different semantic search techniques varied, with IS-BERT providing better results than other techniques, but none showed substantial improvement. The performance of the model is not seriously affected by the various input patterns. The study also discovered that ELECTRA performed better on this task than BERT.

One possible explanation for the lack of improvement after adding context using party manifestos is the models' inability to handle the added complexity or the presence of ambiguous language in the manifestos. The study only examined political party manifestos written in German, so it's important to keep in mind that the findings might not apply to other languages or contexts. Additionally, manifestos might not always include all the necessary information, so expanding the model's data set to include information from other sources like speeches and news articles could increase its accuracy. Better semantic search techniques are also required, especially for German, and testing out various models and hyperparameters may help determine a more effective strategy.